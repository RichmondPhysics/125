%\documentclass[english,twoside]{article}
\documentclass[english,twoside]{labmanual} %labmanual.cls is a modified version of article.cls, 
                                                                     %tweaked to handle \part{} differently

\input{../../131/StudentGuideModule1/labmanual_formatting_commands} %all general latex packages, commands, and definitions now here.

%The \includeonly line below is a great way to save time so you don't always have to compile the WHOLE latex document, if for instance you've only made changes to a single lab.  If you want to compile more than two labs, the syntax is \includeonly{lab1,lab2,lab3} with no spaces after the commas.
%The master.pdf produced will have only the title page, TOC, and that single lab, though the other lab names will appear in the TOC.
%\includeonly{circ_motion/circ_motion}

%Use the following line to override all of the 1's and 0's in the \includelab statements below
%\includealllabstrue

%\renewcommand{\section}{\cleardoublepage\NoExtraPageSection} %Makes each lab start on odd numbered page (right hand side).

\makeindex
\begin{document}

\input{125_front_pages/125_title_page}

\setcounter{tocdepth}{1}\tableofcontents{}

%use \includelab[1] to include, or \includelab[0] to exclude.  Can be overridden with \includealllabstrue
%--------------------------------------------
\part{Kinematics}

\include{measuring}
%\include{walking_speed}
%\include{converting_units}
\include{measurement}
%\include{measurement_length}
%\include{determining_pi}
%\include{measurement_uncertainty}
\include{position}
\include{velocity}
\include{relating}
%\include{instantaneous_velocity}
\include{changing}
\include{slowing}
%\include{equations}
\include{grav_freefall}
%\include{acceleration}
%\include{reaction_time}
%\include{tossed_ball}
%\include{independence}
%\include{ski_jump}
%\include{projectile}
%\include{circ_motion}

%--------------------------------------------
\part{Newton's Laws}

\include{force1}
%\include{force2}
%\include{combining}
%\include{force_mass}
\include{newton}
%\include{atwood}
\include{friction}
\include{elec_grav}
\include{momentum}
%\include{centripetal}
%\include{kepler}

%--------------------------------------------
\part{Conservation Laws}

\include{work_power}
\include{work_kinetic}
\include{conservation_mech}
%\include{conservative}
%\include{momentum}
%\include{impulse}
%\include{newtons_laws}
%\include{mom_cons}
%\include{twod_collisions}

%--------------------------------------------
%\part{Rotational Motion}

%\include{ang_lin_displacement}
%\include{rotation}
%\include{newtons_2law_rot}
%\include{rolling}
%\include{moment_inertia}
%\include{ang_mom}
%
\section{Conservation of Angular Momentum}

Name \rule{2.0in}{0.1pt}\hfill{}Section \rule{1.0in}{0.1pt}\hfill{}Date \rule{1.0in}{0.1pt}

\textbf{Objectives} 

To test the Law of Conservation of Angular Momentum and to explore the applicability
of angular momentum conservation among objects that experience no external torques. 

\textbf{Apparatus}

\begin{itemize}
\item A Rotating Disk System 
\item A mass of 1 kg 
\item A meter stick and a ruler 
\item A small water bubble level
\item A video analysis system (\textit{VideoPoint})
\end{itemize}
\textbf{Overview }

As a consequence of Newton's laws, angular momentum like linear momentum is
believed to be conserved in isolated systems. This means that, no matter how
many internal interactions occur, the total angular momentum of any system should
remain constant if there are no external torques. When one of the objects gains some angular momentum another
part of the system must lose the same amount. If angular momentum isn't conserved,
then we believe that there is some outside torque acting on the system. By expanding
the boundary of the system to include the source of that torque we can always
preserve the Law of Angular Momentum Conservation. 

In this unit you will test the notion of the conservation of angular momentum.
As in the test of the conservation of linear momentum, we will investigate what
happens when two bodies undergo a ``rotational'' collision.
You will drop a large weight onto a rotating disk and determine the moment of
inertia, the angular speed, and finally, the angular momentum of the rotator-disk-weight
system before and after this perfectly inelastic collision.

\textbf{Activity 1: The Moment of Inertia Before and After the Collision}

(a) Calculate the theoretical value of the moment of inertia of the metal disk
using basic measurements of its radius and mass. Be sure to state units and
show the expression you used!
\vspace{5mm}

\( r_{d} =\)  \hfill{}\( M_{d}= \) \hfill{}
\vspace{5mm}

\( I_{d}= \)
\vspace{5mm}

(b)The rotating fixture that holds the disk has a complex shape. We have determined
its moment of inertia without the disk and recorded the result. Record that
value here. Be sure to state units.
\vspace{5mm}

\( I_{f} =\)
\vspace{5mm}

(c) After dropping the weight on the rotating disk, the system will have a new
moment of inertia. Derive a formula for the moment of inertia of a cylindrical-shaped
weight of mass \( m_{w} \) and radius \( r_{w} \) revolving about the origin
at a distance, \( r_{r} \). (You will have to use the parallel axis theorem to do this.)
\vspace{5mm}

\( I_{w} =\)  
\vspace{5mm}

(d) Measure the mass of the weight and use a vernier caliper to measure its diameter.
\vspace{5mm}

\( m_{w} =\)  \hfill{}\(r_{w} =\) \hfill{}
\vspace{5mm}

(e) Come up with a formula for the moment of inertia, $I$, of the whole system
before and after the collision and calculate the moment of inertia before the
collision only. (The moment of inertia after the collision will be determined AFTER you do the experiment.) Don't forget to include the units.
\vspace{5mm}

\( I_{\mbox{\small before}}= \) 
\vspace{5mm}

\( I_{\mbox{\small after}} =\)  
\vspace{5mm}

\textbf{Activity 2: Making a Movie of the Collision} 

(a) Place the video camera about 1 m above the rotator, align the camera with
the center of the rotator using the pendulum, and center the rotator in the
field of view of the camera by viewing it with the \textit{VideoPoint
Capture} software.
Place a ruler of known length in the field of view of the camera and parallel
to one side of the frame. Check that the rotator is flat with the small water-bubble
level.

(b) Give the rotator a push and begin recording its motion with the video camera.
See \textbf{Appendix D: Video Analysis} for details on making the movie. While the rotator is moving
hold the 1 kg weight near the rim of the metal disk and close to, but not quite
touching, the surface of the moving metal disk. After at least one revolution
of the metal disk drop the 1-kg mass onto the disk and record the motion of
the disk for at least one revolution afterward. 

(c) \textit{Before removing the 1 kg weight from the metal disk,} determine the distance of the center of the weight you dropped from the center of the rotator \( r_{r} \). To do this, measure the distance from the center
of the rotator to the edge of the weight \( r_{edge} \) and use the result from Activity 1
part (d) for the diameter of the weight. Calculate the distance from the origin
to the center of the weight \( r_{r} \). Use these results and those from Activity 1 part (e) to calculate the final moment of inertia.
\vspace{5mm}

\( r_{\mbox{\small edge}}  =\) \hfill{}\( r_{w} \) = \hfill{}\( r_{r} \) =\hfill{}
\vspace{5mm}

\( I_{\mbox{\small after}} =\)  
\vspace{5mm}

\textbf{Activity 3: Measurement of Angular Velocity}

Determine the angular speed before and after the collision.

\begin{enumerate}
\item Find the last frame before you dropped the weight on the rotator and click on the position of the white marker on the metal disk. Now go backward through the film until the rotator has gone through one full rotation. Estimate how many frames there are in one rotation. This will be called \( N_{\rm  before}\). You also need to know the time between frames \( \Delta  t_{\rm frame}\), which is just the reciprocal of the capture rate in frames per second.
\[
N_{\mbox{\small before}}=\qquad \qquad \qquad \Delta t_{\rm frame}=\]
Calculate the time for one rotation (the period of rotation) before the collision and the angular speed in radians per second.
\[
T_{\mbox{\small before}}=\qquad \qquad \qquad \omega _{\mbox{\small before}}=\]

\item We now follow a similar procedure to determine the angular speed after the collision. Find the first frame after you dropped the weight on the rotator
and click on the position of the white marker. Now go forward through the film and estimate the number of frames in one full rotation.
\[
N_{\mbox{\small after}}=\qquad \qquad \qquad \Delta t_{\rm frame}=\]
Calculate the time for one rotation after the collision and the angular speed in radians per second.
\[
T_{\mbox{\small after}}=\qquad \qquad \qquad \omega _{\mbox{\small after}}=\]

\end{enumerate}

\vspace{10mm}

(e) Calculate the angular momentum before and after the collision including UNITS. Calculate the percent difference between the two results. Is angular momentum conserved?
\vspace{10mm}

\( L_{\mbox{\small before}}= \)  
\vspace{10mm}

\( L_{\mbox{\small after}}= \)
\vspace{20mm}

%(f) Would the procedure you followed above change if the weight was moving horizontally
%at a constant velocity when you dropped it? If it changed, what would be different?



%--------------------------------------------
\part{Oscillation}

\include{hooke}
\include{periodic_motion}
\include{pendulum_period}
%\include{resonance}
%\include{Standing_waves_strings}
%
\section{Resonance in Tubes}

Name \rule{2.0in}{0.1pt}\hfill{}Section \rule{1.0in}{0.1pt}\hfill{}Date \rule{1.0in}{0.1pt}



{\noindent \bf Objectives:} \begin{list}{$\bullet$}{\itemsep0pt \parsep0pt}

\item Determine the resonant frequency for a tube open at one end.

\item Determine tube lengths at resonance for a tube of variable length.

\item Determine the velocity of sound in air in the laboratory (two ways).

\end{list}

\noindent {\bf Introduction} 

The Economy Resonance Tube is designed for the study of resonance in columns of air.  The tube set includes a movable inner tube with a closed end and an outer tube which is open at both ends.  The inner tube also includes a measuring tape to easily find the length of the air column in the outer tube. To adjust the length of the outer tube, simply slide the inner tube until the desired length appears on the measuring tape.  Open tube experiments can also be performed with the outer tube by removing the inner tube.

\noindent In order that the tube resonate, the frequency of the vibrating air must coincide with the natural frequency of the tube (which may be its fundamental or one of its overtones). For the Economy Resonance Tube, which is closed at one end, this requirement is met if the tube length is an odd number of quarter wavelengths of the sound waves produced by the source ($L = \lambda/4, 3 \lambda/4, 5 \lambda/4$, etc., where $L$ is the length of the tube and $\lambda$ is the wave length of the sound). Note that if the length of the tube is gradually increased while the source is vibrating, the distance between successive resonance positions is $\lambda/2$. \\

\noindent {\small {\bf Note:} Due to edge effects at the open end of a tube, the effective length of the tube depends on the radius of the opening. Thus, $L_{eff} = L + 0.6r$, where $L_{eff}$ is the \textit{effective} length, $L$ is the length measured, and $r$ is the tube radius.} \\


{\noindent \bf Apparatus:} \begin{list}{$\bullet$}{\itemsep0pt \parsep0pt}

\item Economy Resonance Tube 
\item Open speaker
\item Sine wave generator
\item 2 banana plug leads
\item Sound sensor
\item Meter stick
\item Data Studio 750 Interface
\item Thermometer

\end{list}

\noindent Room Temperature ($^\circ$C) \hrulefill \ \  Tube radius (m) \hrulefill

\vspace{5mm}

{\noindent \bf Activity 1: Fixed tube length} \begin{enumerate}

\item Connect the open speaker to the sine wave generator using standard banana plug leads.

\item Adjust the length of the outer tube to 50 cm (check with meter stick).

\item Place the tube in front of the speaker in such a way that the tube is open at one end (the speaker can be set at an angle relative to the tube length).

\item Set the sound sensor inside the tube and connect it to the Data Studio interface.

\item To activate the sound sensor, perform the following sequence:  Start up \textit{DataStudio} by going to \textit{Start} $\rightarrow$ \textit{Programs} $\rightarrow$ \textit{Physics Applications} $\rightarrow$ \textit{DataStudio}.
Click on \textit{Create Experiment}, then \textit{Setup}, then \textit{Add Sensor or Instrument}. Scroll down to \textit{Sound level sensor} and select, then click \textit{OK}. Double click \textit{Graph} at left. Click \textit{Start} to begin taking data.

\item Start at a frequency of 130 Hz and increase until you find the frequency of the largest resonance (indicated by a peak on the sound level graph).  This is the fundamental frequency.  Record the result here:\vspace{10mm}

\item The resonant frequencies for a tube open at one end are given by $f=nv/4L$ where $n$ is an odd integer, $v$ is the velocity of sound and $L$ is the effective tube length.  From the fundamental frequency you just found, calculate the velocity of sound in air (using $n$ = 1) and record it here:\vspace{15mm}

\end{enumerate}

%\noindent {\Large{\bf DATA}} \\

%\noindent Room Temperature ($^\circ$C) \hrulefill \ \  Tube radius (m) \hrulefill

%\begin{center} \begin{tabular}{||c|c|c|c|c|c|c|c|c|c|c|c|c|c|c||} \hline \hline Tuning Fork & \multicolumn{4}{|c|}{First Position of} & \multicolumn{4}{|c|}{Second Position} & \multicolumn{4}{|c|}{Third Position of} & Wave- & Velocity of \\ Frequency, & \multicolumn{4}{|c|}{Resonance, m} & \multicolumn{4}{|c|}{of Resonance, m} & \multicolumn{4}{|c|}{Resonance, m} & length, & Sound in \\ \cline{2-13} Hz & 1 & 2 & 3 & Ave. & 1 & 2 & 3 & Ave. & 1 & 2 & 3 & Ave. & m & air, m/s \\ \hline \hline &&&&&&&&&&&&&& \\ \hline &&&&&&&&&&&&&& \\ \hline \hline \end{tabular} \end{center}


{\noindent \bf Activity 2: Fixed frequency} \begin{enumerate}

\item Adjust the outer tube length to 20 cm.

\item Set the speaker inside the open end of the tube so that it is closed at both ends.

\item Set the sine wave generator frequency to 600 Hz (with low amplitude).

\item Slowly move the inner tube to increase the effective length of the tube.  Record the length of the tube when resonance is achieved. You don't need to use the sound sensor for this. Just listen for maximum loudness.\vspace{10mm}

\item Increase the length of the tube until two more resonance lengths are found for the constant frequency and record them here:\vspace{15mm}

%\item Average your two values to determine your experimental velocity of sound in m/s: \ \ \  \rule{2cm}{1pt}

\item The maxima you have determined are spaced a distance $\lambda/2$ apart, where $\lambda$ is the wavelength.  Find the differences between adjacent resonance lengths and calculate the average of the two values:\vspace{15mm}

\item Find $\lambda$ from your average value of $\lambda/2$ and calculate the velocity of sound in air from $v=f\lambda$.\vspace{15mm}

\item The velocity of sound in air at $0^\circ$C is 331.4 m/s.  The temperature dependence of sound velocity in air is given by $v(T) = 331.4 + 0.6T$, where $T$ is in $^\circ$C and $v$ is in m/s. Calculate an ``accepted'' value of the velocity of sound in air from this formula.

\vspace{15mm}

\item What is the percent difference between your experimental result and the ``accepted'' value?

\end{enumerate}


%--------------------------------------------
\part{Thermodynamics}

\include{heat_temp_int_energy}
%\include{calorimetry}
\include{boyles_law}
\include{calorimetry}
%\include{charles_law}
%\include{P-T}

%--------------------------------------------
\part{Mechanical Waves}

\include{Standing_waves_strings}

\section{Resonance in Tubes}

Name \rule{2.0in}{0.1pt}\hfill{}Section \rule{1.0in}{0.1pt}\hfill{}Date \rule{1.0in}{0.1pt}



{\noindent \bf Objectives:} \begin{list}{$\bullet$}{\itemsep0pt \parsep0pt}

\item Determine the resonant frequency for a tube open at one end.

\item Determine tube lengths at resonance for a tube of variable length.

\item Determine the velocity of sound in air in the laboratory (two ways).

\end{list}

\noindent {\bf Introduction} 

The Economy Resonance Tube is designed for the study of resonance in columns of air.  The tube set includes a movable inner tube with a closed end and an outer tube which is open at both ends.  The inner tube also includes a measuring tape to easily find the length of the air column in the outer tube. To adjust the length of the outer tube, simply slide the inner tube until the desired length appears on the measuring tape.  Open tube experiments can also be performed with the outer tube by removing the inner tube.

\noindent In order that the tube resonate, the frequency of the vibrating air must coincide with the natural frequency of the tube (which may be its fundamental or one of its overtones). For the Economy Resonance Tube, which is closed at one end, this requirement is met if the tube length is an odd number of quarter wavelengths of the sound waves produced by the source ($L = \lambda/4, 3 \lambda/4, 5 \lambda/4$, etc., where $L$ is the length of the tube and $\lambda$ is the wave length of the sound). Note that if the length of the tube is gradually increased while the source is vibrating, the distance between successive resonance positions is $\lambda/2$. \\

\noindent {\small {\bf Note:} Due to edge effects at the open end of a tube, the effective length of the tube depends on the radius of the opening. Thus, $L_{eff} = L + 0.6r$, where $L_{eff}$ is the \textit{effective} length, $L$ is the length measured, and $r$ is the tube radius.} \\


{\noindent \bf Apparatus:} \begin{list}{$\bullet$}{\itemsep0pt \parsep0pt}

\item Economy Resonance Tube 
\item Open speaker
\item Sine wave generator
\item 2 banana plug leads
\item Sound sensor
\item Meter stick
\item Data Studio 750 Interface
\item Thermometer

\end{list}

\noindent Room Temperature ($^\circ$C) \hrulefill \ \  Tube radius (m) \hrulefill

\vspace{5mm}

{\noindent \bf Activity 1: Fixed tube length} \begin{enumerate}

\item Connect the open speaker to the sine wave generator using standard banana plug leads.

\item Adjust the length of the outer tube to 50 cm (check with meter stick).

\item Place the tube in front of the speaker in such a way that the tube is open at one end (the speaker can be set at an angle relative to the tube length).

\item Set the sound sensor inside the tube and connect it to the Data Studio interface.

\item To activate the sound sensor, perform the following sequence:  Start up \textit{DataStudio} by going to \textit{Start} $\rightarrow$ \textit{Programs} $\rightarrow$ \textit{Physics Applications} $\rightarrow$ \textit{DataStudio}.
Click on \textit{Create Experiment}, then \textit{Setup}, then \textit{Add Sensor or Instrument}. Scroll down to \textit{Sound level sensor} and select, then click \textit{OK}. Double click \textit{Graph} at left. Click \textit{Start} to begin taking data.

\item Start at a frequency of 130 Hz and increase until you find the frequency of the largest resonance (indicated by a peak on the sound level graph).  This is the fundamental frequency.  Record the result here:\vspace{10mm}

\item The resonant frequencies for a tube open at one end are given by $f=nv/4L$ where $n$ is an odd integer, $v$ is the velocity of sound and $L$ is the effective tube length.  From the fundamental frequency you just found, calculate the velocity of sound in air (using $n$ = 1) and record it here:\vspace{15mm}

\end{enumerate}

%\noindent {\Large{\bf DATA}} \\

%\noindent Room Temperature ($^\circ$C) \hrulefill \ \  Tube radius (m) \hrulefill

%\begin{center} \begin{tabular}{||c|c|c|c|c|c|c|c|c|c|c|c|c|c|c||} \hline \hline Tuning Fork & \multicolumn{4}{|c|}{First Position of} & \multicolumn{4}{|c|}{Second Position} & \multicolumn{4}{|c|}{Third Position of} & Wave- & Velocity of \\ Frequency, & \multicolumn{4}{|c|}{Resonance, m} & \multicolumn{4}{|c|}{of Resonance, m} & \multicolumn{4}{|c|}{Resonance, m} & length, & Sound in \\ \cline{2-13} Hz & 1 & 2 & 3 & Ave. & 1 & 2 & 3 & Ave. & 1 & 2 & 3 & Ave. & m & air, m/s \\ \hline \hline &&&&&&&&&&&&&& \\ \hline &&&&&&&&&&&&&& \\ \hline \hline \end{tabular} \end{center}


{\noindent \bf Activity 2: Fixed frequency} \begin{enumerate}

\item Adjust the outer tube length to 20 cm.

\item Set the speaker inside the open end of the tube so that it is closed at both ends.

\item Set the sine wave generator frequency to 600 Hz (with low amplitude).

\item Slowly move the inner tube to increase the effective length of the tube.  Record the length of the tube when resonance is achieved. You don't need to use the sound sensor for this. Just listen for maximum loudness.\vspace{10mm}

\item Increase the length of the tube until two more resonance lengths are found for the constant frequency and record them here:\vspace{15mm}

%\item Average your two values to determine your experimental velocity of sound in m/s: \ \ \  \rule{2cm}{1pt}

\item The maxima you have determined are spaced a distance $\lambda/2$ apart, where $\lambda$ is the wavelength.  Find the differences between adjacent resonance lengths and calculate the average of the two values:\vspace{15mm}

\item Find $\lambda$ from your average value of $\lambda/2$ and calculate the velocity of sound in air from $v=f\lambda$.\vspace{15mm}

\item The velocity of sound in air at $0^\circ$C is 331.4 m/s.  The temperature dependence of sound velocity in air is given by $v(T) = 331.4 + 0.6T$, where $T$ is in $^\circ$C and $v$ is in m/s. Calculate an ``accepted'' value of the velocity of sound in air from this formula.

\vspace{15mm}

\item What is the percent difference between your experimental result and the ``accepted'' value?

\end{enumerate}


%--------------------------------------------
\part{Electricity and Magnetism}

\include{electrostatics}
%\include{ef_equipot_lines}
\include{EF_EP_1}
\include{EF_EP_2}
\include{ohms_law}
\include{magnetism_1}
\include{magnetism_2}
\include{magnetism_3}
\include{magnetic_field_earth}
\include{electromagnetic_induction}
\include{induction2}

%--------------------------------------------
\part{Optics}

\include{refraction_of_light}
\include{refraction_at_spherical_surfaces}
\include{interference_of_light}
\include{diffraction2}
\include{diffraction_grating}
\include{hydrogen}

%--------------------------------------------
%\part{Other Topics}

%\include{galilean_relativity}
%\include{twins_paradox}

%--------------------------------------------
%\part{Appendices}
\immediateaddcontentsline{toc}{part}{Appendices} %Avoids a Roman part number for the Appendices in the toc
\cleardoublepage
\renewcommand{\section}{\NoExtraPageSection} %Makes NO extra page break after each appendix; Appendices can start on odd or even page (right or left side).
\appendix

\include{treatment_data}
\include{datastudio}
\include{excel}
\include{video_analysis}
\include{video_analysis_tracker}
\include{instrumentation}


\end{document}
