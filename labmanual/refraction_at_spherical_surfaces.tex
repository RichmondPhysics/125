
\section{Refraction at Spherical Surfaces: Thin Lenses}

Name \rule{2.0in}{0.1pt}\hfill{}Section \rule{1.0in}{0.1pt}\hfill{}Date
\rule{1.0in}{0.1pt}

\textbf{Objective}

\begin{itemize}
\item To investigate thin lenses.
\end{itemize}
\textbf{Introduction} 

A lens converges or diverges light rays. It is a transparent material
bounded, in the case of thin lenses, by spherical edges. The line
between the centers of curvature of these edges is referred to as
the principal axis. The principal focus is the point on the principal
axis where parallel incident rays converge. The distance from the
lens to this point is known as the focal length. The relation between
the focal distance, $f$, the object distance, $p$, and the image distance,
$q$, is:

\begin{displaymath} \frac{1}{p} + \frac{1}{q} = \frac{1}{f}. \end{displaymath}

\textbf{Apparatus}

\begin{itemize}
\item light fence
\item converging (convex) lens in holder
\item converging and diverging lenses (1 each) without holders
\item optical bench 
\item light source with 12v output power source
\item optical viewing screen (white)
\item plastic ruler
\end{itemize}
%\textbf{Investigation 1: The Converging Lens}

\textbf{Activity 1}

\begin{itemize}
\item Arrange the light source apparatus so that the parallel rays of light
cross a piece of paper. 
\item Place a convex lens (without holder) on the paper perpendicular to the 
central ray.
Outline its position and the path of the rays. Pay particular attention
to the condition near the principal focus.
\item What is the focal length of this lens?\vspace{15mm}
\item Include a sketch of the light rays in your lab book, showing the focal point and focal length of the lens.
\end{itemize}
\textbf{Activity 2 }

\begin{itemize}
\item Place the light source in its bracket at one end of the optical bench. The arrow on the light source will be the object in this investigation. Measure
and record its height, $h_0$.\vspace{10mm}

\item Place a converging lens in its holder on the optical bench 70 cm from the object (this is the object distance $p$). Turn on the light source and position the optical viewing screen on the optical bench so that a sharp image of the arrow appears on the screen. Measure the distance of the screen from the lens. This is the image distance $q$. Also measure the height of the image $h_i$. Record $p$, $q$, and $h_i$ in the first line of the following table.
\end{itemize}
\vspace{0.3cm}
{\centering \begin{tabular}{|c|c|c|c|c|c|}
\hline 
~~~~~~~\( p \)~~~~~~~&
~~~~~~~\( q \)~~~~~~~&
~~~~~~~\( h_{i} \)~~~~~~~&
~~~~~~~\( \frac{h_{i}}{h_{0}} \)~~~~~~~&
~~~~~~~\( \frac{q}{p} \)~~~~~~~&
~~~~~~~\( f \)~~~~~~~\\
\hline
\hline 
&
&
&
&
&
\\
\hline 
&
&
&
&
&
\\
\hline 
&
&
&
&
&
\\
\hline 
&
&
&
&
&
\\
\hline 
&
&
&
&
&
\\
\hline
\end{tabular}\par}
\vspace{0.3cm}

\begin{itemize}
\item Move the lens to create four more object distances of 60, 50, 40, and 30 cm. In each case, measure the image distance and the height of the image and record in the above table. Calculate and record the ratios of the image and object heights, $h_i / h_0$, and the image and object distances, $q / p$. Record these values in the above table. You should now have the first five columns of the above table filled in.
\item Calculate and record the focal length, $f$, for each observation. Show one of the calculations here:\vspace{15mm}
\item Determine an average focal length, $f_{ave}$, and a standard deviation based on your five values.
\vspace{15mm}
\item What is the relationship between the ratio of the image to object
heights and the ratio of image and object distances? The first ratio
is called the magnification.\vspace{15mm}

\item Replace the converging lens with a diverging one. Try to obtain a
real image on a piece of white paper.
\item Why can you not form a real image with a diverging lens?\vspace{15mm}

\end{itemize}
%\textbf{Investigation 2: Lenses in Combination (Optional)}

%\textbf{Activity}

%\begin{itemize}
%\item Place a converging lens and a diverging lens together into the lens
%holder. Check to see that you can get a real image with this combination.
%\end{itemize}
%\vspace{0.3cm}

%\begin{itemize}
%\item Repeat the five sets of observations of Investigation 1, Activity
%2, to get an equivalent focal length $f_{ave}^{eq}$.\vspace{40mm}

%\end{itemize}
%\textbf{Investigation 3: The Diverging Lens}

%\textbf{Activity 1 (Optional)}

%\begin{itemize}
%\item Using the relation:
%\end{itemize}
%\begin{displaymath} \frac{1}{f^{eq}} = \frac{1}{f_1} + \frac{1}{f_2}, \end{displaymath}

%\begin{quote}
%determine the focal length of the diverging lens, $f_2$. Use $f^{eq}_{ave}$
%for $f^{eq}$ and $f_{ave}$ for $f_1$.\vspace{2in}

%\end{quote}
\textbf{Activity 3}

\begin{itemize}
\item Repeat the procedure of Activity 1 with a concave lens. Locate the 
principal focus by extending the refracted rays backwards.

\item What is the focal length of this lens?
\vspace{15mm}

\item Include a sketch of the light rays in your lab book, showing the focal 
point and focal length of the lens.
\end{itemize}

