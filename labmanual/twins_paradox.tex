
\section{The Twins Paradox}

Name \rule{2.0in}{0.1pt}\hfill{}Section \rule{1.0in}{0.1pt}\hfill{}Date
\rule{1.0in}{0.1pt}

\textbf{Objective}

To investigate some of the unusual implications of Einstein's special
theory of relativity.

\textbf{Overview}

Einstein's theory of special relativity leads to a variety of apparent
paradoxes that depart radically from our everyday expectations. One
of the most celebrated is the twins paradox, in which an identical
twin makes a long interstellar journey while the other twin remains
on the (roughly) stationary Earth. When the space-faring twin returns
she finds her partner has aged considerably more than she has. In
this unit you will explore the quantitative aspects of the paradox
and some of the surprising consequences.

\textbf{Activity 1: Setting Things Up}

Problems in special relativity are often very counterintuitive, so
it is instructive to consider the situation non-relativistically first.
Investigate this problem without applying any of the new ideas you
have learned about the theory of special relativity.

One member of a pair of identical twins has decided to embark on a
long space voyage. The two twins have lived their lives in close proximity
to one another and are very similar in appearance. The adventurous
twin boards a fast spacecraft and leaves the Earth behind at a speed
of 0.99c or 99\% of the speed of light. The space-faring twin's itinerary
is rather monotonous, and she simply travels at this constant speed
for a time, turns around, and returns to the Earth at the same speed.
She measures the time of her trip to be \( \Delta  t_{0} \).
In the meantime the Earth-bound twin has seen twenty years pass by.
We will refer to this time as \( \Delta  t\).

(a) In mathematical terms, what is the relationship between the times
\( \Delta  t_{0} \) and \( \Delta t \)?
\vspace{15mm}

(b) Which time is associated with which twin?
\vspace{15mm}

(c) When the twins are reunited will their appearances differ?
\vspace{15mm}

\textbf{Activity 2: Applying Special Relativity}

(a) Now we want to apply the lessons of special relativity. Time dilation
implies that moving clocks run more slowly when observed by someone
in a different inertial frame. For the twins paradox what does this
imply about the time interval the space-faring twin measures during
her trip? Will it be less than, equal to, or greater than the interval
measured by the Earth-bound twin? Will the space-faring twin age more,
less, or the same amount as the Earth-bound twin?
\vspace{35mm}

(b) What is the mathematical relationship between \( \Delta  t_{0} \)
and \( \Delta  t\) according to the special theory of relativity?
\vspace{15mm}

(c) How much time has passed on the Earth-bound twin's clock?
\vspace{15mm}

(d) How much time has passed on the space-faring twin's clock?
\vspace{15mm}

(e) If this result is inconsistent with your prediction above how
should you resolve the contradiction?
\vspace{15mm}

(f) How will the two twins' appearances differ, if at all? Is the
difference only in the measurement of the time intervals or are there
real physiological differences between the twins after the trip?
\vspace{15mm}

(g) If the average speed of the space-faring twin was more like the
typical orbital speed of the space shuttle (about 7.4 km/s) what would
the time difference between the twins' clocks be?
\vspace{15mm}

\textbf{Activity 3: Graphical Analysis}

(a) Find a mathematical relationship for the ratio of the time interval
measured by the space-faring twin to the time interval measured by
the Earth-bound twin. Show your work and record your result here.
\vspace{45mm}

(b) You will now use \textit{Excel} to make a plot showing how this
ratio behaves as a function of $\beta$ (the speed expressed
as a fraction of the speed of light).
To do this, start up \textit{Excel}, and create a column headed
``beta.''  This column should contain the series of numbers
0,0.05,0.10,0.15,$\ldots$,1.    To create
such a column of numbers, enter the first two rows and highlight
them.  Then grab the lower-right corner of the second cell with
the mouse and drag down (in the same way as if you were dragging
down a formula).  

After you have created the $\beta$ column, create a second
column containing the ratio of time intervals (i.e., the
relationship you found in part (a)).  Use an
\textit{Excel} formula.



(c) Make a plot of the ratio of the time interval measured by the
space-faring twin to the time interval measured by the Earth-bound
twin. At what 
speed does the effect of time dilation become significant? Is there
a limit to the ratio? Is there any reason to restrict the range of
\( \beta  \) to 0-1? Clearly state your reasoning. Print your plot
and attach a copy to this unit.
\vspace{10mm}

(d) Consider the following scenario. As the space-faring twin's craft
recedes from the Earth it is moving at a constant speed. Since no
inertial frame can be considered {}``better'' than any other there
is nothing physically inconsistent with the view that the space-faring
twin is observing the Earth recede from her at a constant velocity.
Hence, the space-faring twin will observe clocks on the Earth to move
slowly and the Earth-bound twin will age at a slower rate than the
space-faring one. Is this reasoning flawed? How? 
\vspace{15mm}

(e) If the scenario is not flawed how can it be that the space-faring
twin was found to have aged less in the original problem?
