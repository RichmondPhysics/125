
\section{The Period of a Pendulum}

Name \rule{2.0in}{0.1pt}\hfill{}Section \rule{1.0in}{0.1pt}\hfill{}Date \rule{1.0in}{0.1pt}

{\noindent \bf Objectives:} \begin{list}{$\bullet$}{\itemsep0pt \parsep0pt}

\item Study the influences on the motion of a simple pendulum \item Calculate the acceleration due to gravity from measurements of the period and length of a simple pendulum

\end{list}

{\noindent \bf Introduction:}

\noindent The period of a simple pendulum is related to its length and the acceleration due to gravity according to the relationship:
\[
T=2\pi \sqrt{\frac{L}{g}}.\qquad [Eq.\: 1]\]

\noindent $T$ is the period, $L$ is the length, and $g$ is the gravitational acceleration. This assumes the oscillations are small. Let's check this prediction experimentally. \\

{\noindent \bf Apparatus:} \begin{list}{$\bullet$}{\itemsep0pt \parsep0pt}

\item string attached to stand \item collection of masses \item stop watch \item meter stick

\end{list}

{\noindent \bf Activity:} \begin{enumerate}

\item With $L$ = 1.0 m, place a 1-kg mass at the end of your pendulum. Time twenty-five (25) oscillations of amplitude not greater than 10 degrees. The period is the total time divided by the number of oscillations. Calculate the period and enter the relevant data into the table below.

\item Repeat 1 for masses of 500 g, 200 g, 100 g, 50 g, and 20 g.

\begin{center} \begin{tabular}{|c|c|c|c|c|c|} \hline Trial No. & Mass & Length[$L$] & No. of & Total Time & Period[$T$] \\ & (kg) & (m) & Oscillations & (s) & (s) \\ \hline \hline 1 & & & & & \\ \hline 2 & & & & & \\ \hline 3 & & & & & \\ \hline 4 & & & & & \\ \hline 5 & & & & & \\ \hline 6 & & & & & \\ \hline \end{tabular} \end{center}

\item With the 200-g mass, fix the length $L$ to be 1.5 m. Time twenty-five (25) oscillations of amplitude not more than 10 degrees. Calculate the period and period squared, and enter the relevant data into the table below.

\item Repeat 3 for pendulum lengths of 1.0, 0.7, 0.4, 0.25, and 0.15 meters.

\begin{center} \begin{tabular}{|c|c|c|c|c|c|c|} \hline Trial No. & Mass & Length[$L$] & No. of & Total Time & Period[$T$] & $T^2$ \\ & (kg) & (m) & Oscillations & (s) & (s) & (s$^2$) \\ \hline \hline 7 & & & & & & \\ \hline 8 & & & & & & \\ \hline 9 & & & & & & \\ \hline 10 & & & & & & \\ \hline 11 & & & & & & \\ \hline 12 & & & & & & \\ \hline \end{tabular} \end{center}

\item Plot $T$ as a function of mass from the first set of data and $T^2$ as a function of $L$ from the second set of data on SEPARATE graphs. NOTE: Be sure the $T$ versus mass graph contains the origin. (If you don't know how to do this, consult your instructor.) Fit the data and determine the slopes of the lines of each graph. Be sure to include UNITS with each slope. Print both graphs and include with this unit.

\end{enumerate}

\vspace{10pt}

slope: period versus mass \rule{1.5in}{0.2pt}

\vspace{10pt}

slope: period$^2$ versus length \rule{1.5in}{0.2pt}

\vspace{10pt}

{\noindent \bf Questions:}

\begin{enumerate}
\item Interpret the slope of the period versus mass line: What is the relationship
between mass and period? How does the period depend on the mass? \vspace{20mm}

\item Interpret the slope of the period$^2$ versus length line: What is the relationship
between length and period? How does the period depend on pendulum length? \vspace{20mm}

\item If the length of the pendulum were $\frac{1}{16}$ its original length, by how
much would its period change? \vspace{20mm}

\item Using the relationship between length and period (equation 1) and the slope you measured for the $T^2$ vs $L$ graph, determine the acceleration due to gravity $g$.  Calculate the percent difference between your value and the accepted value of [9.8 $\frac{\rm m}{\rm s^2}$].
\end{enumerate}
