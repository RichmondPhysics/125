
\section{Newton's Second Law for Rotation\footnote{
1990-93 Dept. of Physics and Astronomy, Dickinson College. Supported by FIPSE
(U.S. Dept. of Ed.) and NSF. Portions of this material may have been modified
locally and may not have been classroom tested at Dickinson College.
}}

Name \rule{2.0in}{0.1pt}\hfill{}Section \rule{1.0in}{0.1pt}\hfill{}Date \rule{1.0in}{0.1pt}

\textbf{Objectives} 

To understand torque and its relation to angular acceleration and rotational
inertia on the basis of both observations and theory. 

\textbf{Apparatus}

\begin{itemize}
\item A Rotating Disk System 
\item A hanging mass of 300-400 g (for applying torque) 
\item String and pulley
\item A meter stick and a ruler 
\item A video analysis system (\textit{VideoPoint}).
\end{itemize}
\textbf{Overview} 

We have used the definition of rotational inertia, $I$, to determine a theoretical
equation for the rotational inertia of a disk. This equation was given by
\[
I=\frac{1}{2}Mr^{2}.\]


Does this equation adequately describe the rotational inertia of a rotating
disk system? If so, then we should find that, if we apply a known torque, \( \tau  \),
to the disk system, its resulting angular acceleration, \( \alpha  \), is actually
related to the system's rotational inertia, $I$, by the equation
\[
\tau =I\alpha \quad \mbox{or}\quad \alpha =\tau /I\]


The purpose of this experiment is to determine if, within the limits of experimental
uncertainty, the measured angular acceleration of a rotating disk system is
the same as its theoretical value. The theoretical value of angular acceleration
can be calculated using theoretically determined values for the torque on the
system and its rotational inertia.

\textbf{Theoretical Calculations} 

You'll need to take some basic measurements on the rotating disk system
to determine theoretical values for $I$ and \( \tau  \). Values of rotational
inertia calculated from the dimensions of a rotating object are theoretical
because they purport to describe the resistance of an object to rotation. An
experimental value is obtained by applying a known torque to the object and
measuring the resultant angular acceleration.

\textbf{Activity 1: Theoretical Calculations }

(a) Calculate the theoretical value of the rotational inertia of the metal disk
using basic measurements of its radius and mass. Ignore the small hole in the
middle in your calculation. Be sure to state units!
\vspace{5mm}

\( r_{d} =\) \hfill{}\( M_{d}= \) \hfill{}
\vspace{5mm}

\( I_{d}= \) 
\vspace{5mm}

(b)The rotating fixture that holds the disk has a complex shape. We have determined
its moment of inertia without the disk and recorded the result on the fixture.
Record that value here. Be sure to state units.
\vspace{5mm}

\( I_{f} =\) 
\vspace{5mm}

(c) Calculate the theoretical value of the rotational inertia, $I$, of the whole
system. Don't forget to include the units.
\vspace{5mm}

$I =$
\vspace{5mm}

(d) In preparation for calculating the torque on your system, summarize the
measurements for the falling mass, $m$, and the radius of the spool that has the
string wrapped around it in the space below. Don't forget the units!
\vspace{5mm}

$m = $\hfill{}\(r_{s}= \)\hfill{} 
\vspace{5mm}

(e) Calculate the theoretical value for the torque on the rotating system as
a function of the magnitude of the hanging mass and the radius, \( r_{s} \),
of the spool, assuming the tension in the string is equal to the weight of the falling mass (this introduces an error of less than 1 percent).  Be sure to include units.
\vspace{5mm}

\( \tau _{th}= \)
\vspace{5mm}

(f) Based on the values of torque and rotational inertia of the system, what
is the theoretical value of the angular acceleration of the disk? What are the
units? 
\vspace{5mm}

\( \alpha _{th}= \)
\vspace{5mm}

\textbf{Activity 2: Experimental Measurement of Angular Acceleration} 

(a) Place the video camera about 1 m above the rotator, and center the rotator in the field of view of the camera by viewing the rotator with the \textit{VideoPoint Capture} software. Use the small level to ensure that the surface of the rotator is level. Place a ruler of known length in the field of view of the camera and parallel to one side of the frame.

(b) Place the rotator so the string will pass smoothly over the pulley and put
300-500 g of mass on the end of the string. Release the rotator and use the
video camera to record the motion of the disk for at least two full turns. See
\textbf{Appendix D: Video Analysis} for details.

(c) Determine the angular displacement of the rotator as a function of time.
Be careful to place the origin of your coordinate system on the axle of the
rotator so the angular displacement you measure will be the desired one. To
do this task follow the instructions in \textbf{Appendix D: Video Analysis}
for recording, calibrating, and analyzing a movie data file. The file should contain three columns with the values of time, x-position, and y-position for one complete revolution.

(d) What is the expression for the angular displacement of the disk in terms
of the x and y positions of the marker that you recorded above? Note that these
positions should be relative to an origin placed on the axle of the rotator.
\vspace{5mm}

\( \theta  \)= 
\vspace{5mm}

(e) We want to graph the angular displacement of the disk as a function of time.
To do this:

\begin{enumerate}
\item Export your data to an \textit{Excel} file and launch 
\textit{Excel}.

\item Calculate the angular displacement \( \theta  \) in radians for the first row
in the spreadsheet. Record the result here.
You will use this result later to check the calculations you make with 
\textit{Excel}.


$\theta =$

\item Calculating the angular position for all the data as we just did
would be horribly tedious. Instead, use an {\it Excel} formula to
figure out the angular positions.  (See Appendix C for details.)
You may find it helpful to know that there is an {\it Excel} function
ATAN2 that takes the inverse tangent of the ratio of two numbers.
For instance, if you put ``=ATAN2(C3,D3)'' into a cell, {\it Excel}
will calculate the inverse tangent of the ratio of the number
in cell D3 to the number in cell C3.  (Note that the ratio
that is taken has the second argument on top: in this case, it's 
D3/C3, not C3/D3.)

\item Does the value of the first row agree with the calculation you made in part
(2) above? If it does not check both calculations again. If that fails consult
your instructor. 
\item Graph the angular displacement (column 4) as a function of time (column 1).
You will see discontinuous jumps in your data because the function you used
in part (3) always calculates angles in the range \( -\pi  \) to \( \pi  \).
You must add different increments of \( 2\pi  \), \(4 \pi  \), etc. to adjust
the scale of the angular displacement.  You should create another
column of data in your data table containing the angular positions with
appropriate multiples of $2\pi$ added to them.
\end{enumerate}
(f) We now want to extract the angular acceleration from the data.

\begin{enumerate}
\item To describe the time dependence of the angular displacement what type of polynomial
should we use to fit the data? How are the coefficients of the polynomial related
to the angular acceleration?\vspace{20mm}

\item Fit the data with a polynomial and write the resulting equation for the time
dependence of the angular position in the space below. Be sure to include the
proper units with the coefficients. Determine the experimental value for the
angular acceleration from the fit and record it below. Also, print a copy of
your plot with the fitted curve and the equation and put it in your notebook.\vspace{20mm}

\end{enumerate}

\newpage

Compare your experimental results for \( \alpha \) to your theoretical calculation of \( \alpha \) for the rotating system. Present this comparison with a summary of your data and calculated results.

\textbf{Activity 3: Comparing Theory with Experiment }

(a) Summarize the theoretical and experimental values of angular acceleration.
\vspace{5mm}

\( \alpha _{th}= \)
\vspace{5mm}

\( \alpha _{exp}= \) 
\vspace{15mm}

(b) Do theory and experiment agree within the limits of experimental uncertainty?
What is the percent deviation?

